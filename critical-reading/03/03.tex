\documentclass{math174}

\criticalreading{2}
\sections{1.6}
\problems{(1.13) 8, 10}
\date{Monday, February 11}
\author{}
\coauthor{}

\begin{document}
\begin{description}
\item[1.13.8] Let \(V\) be a vector space.  Show that the following
  properties will hold in \(V\) if and only if a related property
  holds for a basis of \(V\).
  \begin{enumerate}
  \item \(V\) is a \(G\)-module.
  \item The map \(\theta \colon V \to W\) is a \(G\)-homomorphism.
  \item The inner product \(\inner{\cdot, \cdot}\) on \(V\) is
    \(G\)-invariant.
  \end{enumerate}

  \begin{solution}

  \end{solution}

\item[1.13.10] Verify that the map \(X \colon \mathbf R^+ \to GL_2\)
  given in the example at the end of Section 1.5 is a representation
  and that the subspace \(W\) is invariant.

  \begin{book}
    \dots{} Let \(\mathbf R^+\) be the positive real numbers, which
    are a group under multiplication.  It is not hard to see that
    letting
    \[
      X(r) =
      \begin{pmatrix}
        1 & \log r \\
        0 & 1
      \end{pmatrix}
    \]
    for all \(r \in \mathbf R^+\) defines a representation.  The
    subspace
    \[
      W = \Setst{
        \begin{pmatrix}
          c \\ 0
        \end{pmatrix}
      }{c \in \CC} \subset \CC^2
    \]
    is invariant under the action of \(G\).
  \end{book}

  \begin{solution}

  \end{solution}
\end{description}
\end{document}