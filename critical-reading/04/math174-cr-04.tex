\documentclass{math174}

\criticalreading{4}
\date{Monday, February 20}

\sections{
  1.7 and
  \href{https://www.todos-math.org/assets/documents/TEEMv1n1excerpt.pdf}
  {Framing Equity}
  by Rochelle Guti\'errez
}

\problems{(1.13) 12, 14; \textit{Framing Equity} etc.}
\plainliststyles

\author{}
\coauthor{}

\begin{document}

\begin{enumerate}
\item
  \begin{description}
  \item[(1.13) 12] Let \(X\) be an irreducible matrix representation
    of \(G\).  Show that if \(g \in Z_g\) (the center of \(G\)), then
    \(X(g) = cI\) for some scalar \(c.\)

    \begin{solution}

    \end{solution}
  \end{description}
\item
  \begin{description}
  \item[(1.13) 14] Prove the following converse of Schur's lemma.  Let
    \(X\) be a representation of \(G\) over \(\CC\) with the property
    that only scalar multiples \(cI\) commute with \(X(g)\) for all
    \(g \in G\).  Prove that \(X\) is irreducible.

    \begin{solution}

    \end{solution}
  \end{description}
\item In \textit{Framing Equity}, what are the dimensions of equity
  describedby Guti\'errez?

  \begin{solution}

  \end{solution}

\item For each dimension of equity, give an example from your lived
  experience which either reinforces Guti\'errez's framework or
  complicates/challenges it.

  \begin{solution}

  \end{solution}
\end{enumerate}




\end{document}