\documentclass{math174}

\homework{1}
\problems{(1.13) 1}

\date{Wednesday, January 30}
\author{}
\coauthor{}

\begin{document}

\begin{problem}[1.13.1]
  An \emph{inversion} in \(\pi = x_1, x_2, \dots, x_n \in S_n\)
  (one-line notation) is a pair \(x_i, x_j\) such that \(i<j\) and
  \(x_i>x_j\).  Let \(\inv\pi\) be the number of inversions of
  \(\pi\).
  \begin{enumerate}
  \item \label{part:k-inv-pi} Show that if \(\pi\) can be written
    as a product of \(k\) transpositions, then
    \(k \equiv \inv\pi \pmod 2\).
    \begin{solution}
    \end{solution}
  \item Use part \ref{part:k-inv-pi} to show that the sign of \(\pi\)
    is well-defined.
    \begin{solution}
    \end{solution}
  \end{enumerate}
\end{problem}

\end{document}